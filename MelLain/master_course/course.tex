\documentclass[12pt]{article}
\usepackage[utf8]{inputenc}
\usepackage[russian]{babel}
\usepackage{graphicx}
\usepackage{amsmath}
\usepackage{hyperref}
\usepackage{algorithm}
\usepackage{cmap}  
\usepackage{epstopdf}
\usepackage[noend]{algorithmic}

\textheight = 24.5cm
\textwidth = 17cm
\oddsidemargin = 0pt
\topmargin = -2.2cm
\parskip = 5pt
\tolerance = 2000
\hyphenpenalty = 10


\def\algorithmicrequire{\textbf{Вход:}}
\def\algorithmicensure{\textbf{Выход:}}
\def\algorithmicif{\textbf{если}}
\def\algorithmicthen{\textbf{то}}
\def\algorithmicelse{\textbf{иначе}}
\def\algorithmicelsif{\textbf{иначе если}}
\def\algorithmicfor{\textbf{для}}
\def\algorithmicforall{\textbf{для всех}}
\def\algorithmicdo{\textbf{выполнять}}
\def\algorithmicwhile{\textbf{пока}}
\def\algorithmicrepeat{\textbf{повторять}}
\def\algorithmicuntil{\textbf{пока}}
\def\algorithmicloop{\textbf{цикл}}
\def\algorithmiccomment#1{\quad// {\sl #1}}


\begin{document}

\begin{center}
\thispagestyle{empty}
Московский Государственный Университет им. М.В. Ломоносова

\begin{figure}[h!]
\begin{center}
\includegraphics[scale = 0.1]{title_image.png}
\end{center}
\end{figure}

Факультет Вычислительной Математики и Кибернетики

Кафедра Математических Методов Прогнозирования

\vspace{88pt}
\large

\LARGE
{\bf Курсовая работа}
\vspace{20pt}

{\bf <<Регуляризация тематических моделей в библиотеке BigARTM>>}

\vspace{120pt}
\begin{flushright}
\normalsize

\hfill\parbox{8cm}{
Выполнил:

\vspace{5pt}
студент 3 курса 317 группы

\vspace{5pt}
{\it Апишев~Мурат~Азаматович}

\vspace{20pt}
Научный руководитель:

\vspace{5pt}
д.ф-м.н., доцент

\vspace{5pt}
{\it Воронцов~Константин~Вячеславович.}
}

\vspace{80pt}
\center{Москва, 2014}
\end{flushright}

\newpage
\end{center}

\tableofcontents
\newpage

\section{Введение}\label{introduction}

$\quad\;\:$Тематическое моделирование --- активно развивающаяся в последние годы область машинного обучения. Оно позволяет решать задачи тематического поиска, категоризации и кластеризации корпусов текстовых документов. Аналогичные задачи решаются для коллекций изображений и видеозаписей.

Тематическая модель определяет, к каким темам относится каждый документ, а также то, какие термины из словаря образуют ту или иную тему.

Одной из проблем существующих алгоритмов обучения является невозможность совмещения большого числа функциональных требований в одной модели. LDA, который является на сегодняшний день стандартом в данной области, не позволяет комбинировать более 2-3 требований из-за сложности используемого математического аппарата.

В работе \cite{voron2013ptm} рассматривается теория аддитивной регуляризации тематических моделей (АРТМ), построенная над более простым алгоритмом обучения PLSA, и позволяющая решить описанную проблему. Библиотека BigARTM представляет собой программную реализацию этой концепции, адаптированную под мультипроцессорное и кластерное распараллеливание.

Список основных задач данной курсовой работы: произвести обзор BigARTM, реализовать в ней  механизм добавления/удаления регуляризаторов, описать возможности для его использования и модификации, оценить качество работы. 

{\bf Замечание:} Поскольку библиотека в настоящий момент находится в стадии разработки, описанные возможности и API являются текущими и не отражают полной картины того, как будет выглядеть BigARTM в своей release-версии.

\section{Терминология}\label{defenitions}

Прежде всего рассмотрим некоторые базовые понятия и необходимые обозначения.

Вероятностная тематическая модель (ВТМ) описывает каждую тему дискретным распределением на множестве терминов, каждый документ --- дискретным распределением на множестве тем. Предполагается, что коллекция документов --- это последовательность терминов, выбранных случайно и независимо из смеси таких распределений, и ставится задача восстановления компонент смеси по выборке.

\begin{itemize}
	\item $D$ --- коллекция текстовых документов.
	\item $W$ --- словарь коллекции текстов.
	\item $T$ --- множество тем.
\end{itemize}

Документы в коллекции можно представить в виде так называемого <<мешка слов>>. В рамках этой концепции документ рассматривается как множество терминов из словаря и соответствующих им счётчиков частот встречаемости.

При рассмотрении коллекции в виде пар $(d, w)$, где $w$ --- номер термина, а $d$ --- номер документа, вводятся следующие счётчики частот:

\begin{itemize}
	\item $n_{dw}$ --- число вхождений термина $w$ в документ $d$;
	\item $n_d = \sum_{w \in W} n_{dw}$ --- длина документа $d$ в терминах;
	\item $n_w = \sum_{d \in D} n_{dw}$ --- число вхожденией документа $w$ во все документы коллекции;
	\item $n = \sum_{d \in D}\sum_{w \in d} n_{dw}$ --- длина коллекции $D$ в терминах; 
\end{itemize}

Если же рассматривать коллекцию в виде троек $(d, w, t)$, где $d$, $w$ и $t$ --- номера соответствующих документа, термина и темы, то можно ввести такие счётчики: 

\begin{itemize}
	\item $n_{dwt}$ --- число троек, в которых термин $w$ встретился в документе $d$ и связан с темой~$t$;
	\item $n_{dt} = \sum_{w \in W} n_{dwt}$ --- число троек, в которых термин из документа $d$ связан с темой $t$;
	\item $n_{wt} = \sum_{d \in D} n_{dwt}$ --- число троек, в которых термин $w$ связан с темой $t$;
	\item $n_t = \sum_{d \in D}\sum_{w \in d} n_{dwt}$ --- число троек, связанных с темой $t$;
\end{itemize}

С использованием данных счётчиков можно ввести следующие частотные оценки вероятностей, связанных со скрытой переменной $t$:

\begin{itemize}\label{label_1}
\item $ 
	\hat p(w|t) = \cfrac{n_{wt}}{n_t}, \quad
	\hat p(t|d) = \cfrac{n_{dt}}{n_d}, $ 
\end{itemize}

Вводятся следующие матрицы:
\begin{itemize}
\item $\Phi = (\phi_{wt})_{W \times T}, \; \phi_{wt} = \hat p(w|t)$ --- матрица <<термины-темы>>.
\item $\Theta = (\theta_{td})_{T \times D}, \; \theta_{td} = \hat p(t|d)$ --- матрица <<темы-документы>>. 
\item $F$ --- заданная матрица частот <<термины-документы>>, $F \approx \Phi \Theta$.
\end{itemize}

Задача тематического моделирования --- найти матрицы $\Phi$ и $\Theta$, максимизирующие следующий функционал:

\begin{equation}\label{eq_1}
 	L(\Phi, \Theta) = \sum_{d \in D} \sum_{w \in d} n_{dw} \sum_{t \in T} \phi_{wt} \theta_{td} \rightarrow \max_{\Phi, \Theta}
\end{equation}


\section{Обучение тематических моделей}\label{learning}

\subsection{PLSA}

Вероятностный латентный семантический анализ (PLSA) был предложен Т.Хофманном в~\cite{hofmann_plsa}.

Примем гипотезу условной независимости, утверждающую, что вероятность появления термина в данном документе зависит только от темы этого термина и не зависит от документа. Вероятностная порождающая модель PLSA имеет следующий вид: 
\[
	p(w|d) = \sum_{t \in T} p(w|t) p(t|d)
\]

PLSA можно реализовать с помощью ЕМ-алгоритма. Итерационный процесс состоит из двух шагов --- Е-шага (Expectation) и М-шага (Maximization). На Е-шаге по текущим значениям $\phi_{wt}$ и $\theta_{td}$ c помощью формулы Байеса вычисляются условные вероятности $p(t|d,w)$:
\[
	H_{dwt} = p(t|d,w) = \cfrac{\phi_{wt}\theta_{td}}{\sum_{s \in T}\phi_{ws}\theta_{sd}}
\]

На М-шаге по условным вероятностям $H_{dwt}$ вычисляются новые приближения параметров $\phi_{wt}$ и $\theta_{td}$. Используются указанные в предыдущем разделе формулы:
\[
	\phi_{wt} = \cfrac{n_{wt}}{n_t}, \quad
	\theta_{td} = \cfrac{n_{dt}}{n_d}, \quad	
\]

\subsection{Аддитивная регуляризация}

Неоднозначность матричного разложения $F \approx \Theta \Phi$ даёт свободу выбора матриц из правой части равенства, позволяя наложить на тематическую модель дополнительные требования.  
Модифицируем максимизируемый функционал \ref{eq_1}:

\begin{equation}
	\quad L(\Phi,\Theta) + R(\Phi,\Theta) \rightarrow \max_{\Phi,\Theta}
\end{equation}	


\[
 	R(\Phi,\Theta) = \sum_{i = 1}^{n} \tau_i R_i(\Phi,\Theta)
\]	
где $R_i(\Phi,\Theta)$ --- дополнительные требования к модели, $\tau_i$ --- неотрицательные  коэффициенты регуляризации, выполнены условия неотрицательности и нормировки столбцов матриц $\Phi$ и $\Theta$.
 	 
Решение этой задачи приводит к обощению формул М-шага в ЕМ-алгоритме:
\begin{equation}
	\phi_{wt} = \cfrac{\left(n_{wt} + \phi_{wt} \cfrac{\partial R}{\partial \phi_{wt}} (\Phi,\Theta) \right)_+}{\sum_{u \in W} \left(n_{ut} + \phi_{ut} \cfrac{\partial R}{\partial \phi_{ut}} (\Phi,\Theta) \right)_+}, \quad 
 	\theta_{td} = \cfrac{\left(n_{dt} + \theta_{td} \cfrac{\partial R}{\partial \theta_{td}} (\Phi,\Theta) \right)_+}{\sum_{s \in T} \left(n_{ds} + \theta_{sd} \cfrac{\partial R}{\partial \theta_{sd}} (\Phi,\Theta) \right)_+}
\end{equation} 
 	 
 	 $n_{wt}$ и $n_{dt}$ определяются аналогично из формул предыдущего раздела.
 	 
Таким образом, суть добавления регуляризаторов --- в простом изменении формул М-шага.

{\bf Замечание:} Использование регуляризаторов требует аккуратного выстраивания т.н. траектории регуляризации. Этот процесс включает в себя настройку параметров, определение времени подключения/отключения того или иного регуляризатора, используемого в модели и т.п.

\section{Существующие реализации параллельных алгоритмов}\label{overview}

$\quad\;\:$В данном разделе кратко рассматриваются некоторые известные параллельные алгоритмы тематического моделирования. Все они основаны на модели LDA, имеют разные технические и алгоритмические детали. Рассмотрим их основные преимущества и недостатки, после чего опишем на базе полученных выводов ключевые требования к BigARTM. 

\subsection{Краткий обзор}

\subsubsection{AD-LDA}
$\quad\;\:$Данный алгоритм был предложен в работе \cite{ad_lda}. В AD-LDA был реализован только межпроцессорный параллелизм, возможность работы на кластере не обсуждалась. 

Представленная архитектура предполагает размещение документов в случайном порядке на каждом из участвующих в вычислениях процессоров. Все процессоры одновременно производят сэмплирование, получая счётчики матрицы $\Theta$. Этот процесс происходит до тех пор, пока последний процессор не закончит работу. После производится шаг синхронизации, на котором счётчики со всех процессоров сливаются вместе. Затем происходит обновление глобальной матрицы $\Phi$. Новая $\Phi$ рассылается каждому процессору, и начинается следующая итерация симплирования. И так до тех пор, пока не выполнится заданный критерий останова.

Данная работа является одной из первых, в которых была осознана возможность параллелизации процеса обучения модели, и в этом её основное достоинство. Но сама реализация имеет ряд серьёзных недостатков:

\begin{enumerate}
	\item Скорость работы алгоритма определяется скоростью самого медленного процессора.
	\item Нагрузка на сеть неравномерна --- высокая во время синхронизации и почти нулевая в процессе сэмплирования. 
	\item Система обладает низкой отказоустойчивостью.
	\item Наличие отдельной матрицы $\Phi$ у каждого процессора приводит к большим затратам памяти.
\end{enumerate}

Как видно, большинство недостатков возникают в виде следствия наличия отдельного шага синхронизации.

\subsubsection{PLDA}
$\quad\;\:$PLDA был описан в статье \cite{plda}. Алгоритм представляет собой модификацию AD-LDA, в которой параллелизм реализован в двух вариантах: с помощью MPI и Map-Reduce (при этом работа опять же распределяется между процессорами, кластерная реализация отсутствует). 

Архитектура не претерпела больших изменений. Существенной модификацией является использование контрольных точек --- откачки данных после очередной итерации сэмплирования на жёсткий диск. В случае сбоя имеется возможность восстановить данные и продолжить вычисления. В MPI-реализации это действительно работает хорошо. В Map-Reduce версии реализация контрольных точек требует больших затрат памяти. Авторы сами рекомендуют использовать MPI.

Собственно, помимо низкой отказоустойчивости, PLDA сохранил все недостатки AD-LDA.

\subsubsection{Y!LDA}
$\quad\;\:$Следующим шагом для параллельных алгоритмов обучения тематических стал Y!LDA, описанный в \cite{y_lda}. Авторы рассмотрели недостатки AD-LDA и предложили свой вариант параллелизма, теперь уже как в рамках одной машины, так и на кластере. Основным достижением стало избавление от выделенного шага синхронизации. Авторы отметили его основные недостатки (плохая отказоустойчивость, неравномерная нагрузка на сеть и ожидание медленных процессоров) 
и предложили способ их разрешения, который будет рассмотрен далее.

Коллекция делится между машинами. В свою очередь, в рамках каждой ноды документы распределяются по процессорам. Все процессоры одного компьютера имеют общую матрицу $\Phi$, что существенно снижает затраты памяти. На каждой ноде имеется т.н. поток синхронизации, к которому ядра обращаются независимо по мере завершения обработки очередной порции документов.

Глобальная матрица $\Phi$ хранится в \verb|memcached|
\footnote{memcached — связующее программное обеспечение, реализующее сервис кэширования данных в оперативной памяти на основе парадигмы хеш-таблицы.}
.
Работа кластера организована с использованием т.н. <<архитектуры классной доски>>. Суть её состоит в том, что каждый компьютер в сети обращается к глобальной матрице $\Phi$ независимо от других, обновляя полученные им счётчики. Порцией обновления при этом является один термин. Кластерный параллелизм был реализован в Hadoop.

\subsubsection{PLDA+}
$\quad\;\:$В публикации \cite{plda_plus} была предпринята попытка модифицировать PLDA. Авторы сделали алгоритм параллельным в кластерном смысле. Для того, чтобы побороть проблемы, связанные с шагом синхронизации, было решено использовать конвейерную архитектуру. Для этого всё множество процессоров делится на две группы --- вычислители и коммутаторы. Первая группа производит сэмплирование (собственно расчёт модели), вторая отвечает за своевременную доставку данных. Вкупе с равномерным распределением коллекции в виде блоков по процессорам-вычислителям, а также наличием механизма приоритетов, сглаживающем узкие места во время счёта, это даёт, согласно экспериментам, хорошие результаты. Тем не менее, один существенный недостаток у данной реализации имеется --- значимая часть вычислительных ресурсов тратится не на счёт, а на своевременную доставку данных.

\subsection{BigARTM}
$\quad\;\:$Из всего описанного были сделаны следующие выводы:

\begin{enumerate}
	\item Алгоритм обязательно должен быть асинхронным, что позволит избежать недостатков AD-LDA.
	\item Матрица $\Phi$ должна быть локальной в рамках компьютера, а не ядра, что приведёт к существенной экономии оперативной памяти.
	\item Использование \verb|memcached| является хорошим решением для хранения глобальной матрицы $\Phi$
	\footnote{К сожалению, сервис memcached предназначен для использования только под Linux. Тем не менее, это не является большой проблемой, поскольку большинство кластеров работает именно под Unix-подобными ОС и кластерный параллелизм BigARTM будет ориенторован именно под них. Windows-версия библиотеки, по сути, предназначена для использования на локальной машине. Хотя, возможно, к релизу для Windows-версии будет написан сервис, осуществляющий функции memcached, как это уже сделано в рамках многопроцессорного параллелизма.}
	.
	\item Коллекцию следует разделить на пакеты документов, обрабатываемых независимо.
\end{enumerate}

Все эти идеи положены в основу BigARTM. Они либо уже реализованы, либо будут реализованы к моменту релиза библиотеки.
Таким образом, BigARTM архитектурно похожа на Y!LDA, за исключением использованного фреймворка для организации кластерного параллелизма (MPI вместо Hadoop).

\section{Библиотека BigARTM}\label{library}
BigARTM --- библиотека тематического моделирования, реализующая концепцию аддитивной регуляризации. На данный момент библиотека поддерживает многопроцессорный паралеллизм. Основным алгоритмом является описанный в \ref{plsa_alg} Online Batch PLSA.

\paragraph{Protocol buffers}
\footnote{Подробнее о том, что такое Google Protocol Buffers и как с ними работать, можно прочесть в \cite{protobuf}}
Рассмотрим кратко эту технологию, поскольку в BigARTM она используется очень интенсивно. Protocol Buffers позволяет описывать proto-сообщения на псевдоязыке, которые затем можно преобразовать специальным компилятором (protoc) в структуры данных со всеми необходимыми методами на С++, Python и Java. Ключевой особенностью является возможность обмена сообщениями между этими языками программирования посредством механизма serializer/deserializer, переводящего сообщения в байт-массив и обратно. Данное решение является максимально переносимым и унифицированным.

Все конфигурации библиотеки, такие как настройки тематической модели, параметры регуляризаторов и т.п., определяются и передаются только посредством proto-сообщений. Результирующая модель также описывается соответствующим сообщением. 

{\bf Замечание:} Все необходимые для работы с библиотекой сообщения будут описаны при дальнейшем изложении.

\subsection{Краткое описание}

\paragraph{Представление коллекции} Коллекция представляется в виде <<мешка слов>> (Bag of words). Поскольку библиотека предназначена для обработки больших массивов текстовой информации, она является онлайновой. Механизм потоковой загрузки и обработки информации реализуется разделением коллекции на пакеты (\verb|Batch|), которые обрабатываются по очереди. Каждый \verb|Batch| имеет свой словарь, который может быть как локальным (что предпочтительнее), так и совпадающим со всем словарём коллекции. 

Приведём здесь соответствующее proto-сообщение:

\vspace{4pt}
\noindent
\verb|message Batch {| \\
\verb|  repeated string token = 1;| \\
\verb|  repeated Item item = 2;| \\
\verb|}|
\vspace{4pt}

Первое поле --- набор терминов (словарь), второе --- набор документов (см. \ref{item_label}).

{\bf Замечание:} В силу внутренних особенностей работы BigARTM, оптимальное число \verb|Batches| --- в 4-5 раз больше, чем число используемых процессоров. Этот параметр не повлиет на качество результирующей модели, но скажется на производительности.

\paragraph{Представление документов}\label{item_label}
 В BigARTM реализована гибкая концепция представления данных. Каждый документ является экземпляром класса \verb'Item'. Этот объект, помимо самого документа (хранимого в виде последовательности терминов и их счётчиков), \verb'Item' может содержать произвольные метаданные, связанные с ним. К таким данным относятся информация об авторе, дате публикации, ссылках на документ и из него, тегах и т.п. Всё это может оказаться крайне полезным при использовании тех или иных регуляризаторов.
 
 Рассмотрим proto-сообщение для \verb|Item|
   
 \vspace{4pt}
 \noindent
 \verb|message Item {| \\
 \verb|  optional int32 id = 1;| \\
 \verb|  repeated Field field = 2;| \\
 \verb|}|
 \vspace{4pt} 
 
 Здесь \verb|id| --- идентификатор документа, второе поле --- набор всевозможных объектов в тексте. Объектом может быть сам текст документа, строка с авторами, список тегов и т.п. Работа механизма становится ясной, если взглянуть на определение сообщения для \verb|Field|
 
 \vspace{4pt}
 \noindent
 \verb|message Field {| \\
 \verb|  optional string field_name = 1 [default = "@body"];| \\
 \verb|  repeated int32 token_id = 2;| \\
 \verb|  repeated int32 token_count = 3;| \\
 \verb|}|
 \vspace{4pt} 
 
 Первое поле сообщает о том, что это за объект (по-умолчанию --- основной текст документа). Второе поле содержит список идентификаторов терминов, встречающихся в этом тексте. В третьем находятся соответствующие этим терминам счётчики встречаемости в этом документе.

\paragraph{Входные и выходные данные}
На вход функциям библиотеки могут подаваться как исходные тексты, так и коллекция в виде пакетов. Это крайне удачное решение, так как, во-первых, коллекция в пакетном виде занимает более чем в два раза меньше памяти
\footnote{Никакой архивации при этом не производится. Каждый Batch представляет собой набор Item в виде сериализованных proto-сообщений. Это было сделано из соображений удобства использования, а объём уменьшается в виде побочного эффекта.}
, чем в исходном текстовом, во-вторых, библиотеке не придётся производить процедуру преобразования, что сэкономит время. 
На выходе пользователь получает собственно тематическую модель (матрицу $\Phi$).

\paragraph{Особенности архитектуры} Рассмотрим архитектуру библиотеки на концептуальном уровне.
Для каждого участвующего в работе алгоритма процессора создаётся соответствующий ему объект класса \verb|Processor|. Процессоры взаимодействуют независимо друг от друга. Задача процессора --- производить собственно вычисления. У процессорного элемента имеется своя копия матрицы $\Phi$, которую он использует. Также у каждого \verb|Processor| имеется т.н. очередь процессора, в которую поступают пакеты данных. Процессор анализирует их, выводя матрицу $\Theta$ (сама матрица при этом явным образом в памяти не хранится). После  полученные результаты (счётчики статистик) отправляются в виде блоков в т.н. очередь слияния. Очередь слияния --- это место, откуда получает информацию экземпляр класса \verb|Merger|. Этот объект отвечает за объединение результатов, полученных от всех процессоров, в одно целое и обновление матрицы $\Phi$. Эта матрица используется в работе процессоров. При этом хранится две копии матрицы $\Phi$ --- новая и старая. Старая матрица является доступной для чтения всем процессорам, новая доступна в режиме read-write только \verb|Merger|. В неё записываются обновления статистик, полученные во время очередного прохода по коллекции. После окончания прохода старая матрица замещается новой (операция дешёвая, происходит простое копирование указателей), а под следующую новую матрицу выделяется память.

\paragraph{Ключевые отличия и заимствования}
Обращаясь к разделу \ref{overview}, сравним кратко архитектуру BigARTM с существующими реализациями.
Некоторые идеи, использованные при создании BigARTM, были позаимствованы из ранних работ по схожей тематике~(\cite{smola}, \cite{ad_lda}).
сравнения не было, но будет, как только получим кластерное рапраллеливание

Далее будут рассмотрены основные шаги по настройке и использованию существующей версии BigARTM, а также список нововведений, которые будут в неё внесены в будущем (либо уже внесены, но требуют доработки).

{\bf Замечание:} На данный момент единственным поддерживаемым библиотекой языком является Python, все выкладки будут производится на нём.

{\bf Замечание:} Здесь и далее под внешней итерацией работы алгоритма понимается один проход по всей коллекции, а под внутренней --- один проход по одному документу (\verb|Item|). 

\subsection{Установка}

Для установки библиотеки необходимо выполнить следующую последовательность действий:

\begin{enumerate}
	\item Установить и распаковать boost 1.55, после чего присвоить системной переменной \verb|BOOST\_ROOT| путь к корню распакованной папки (если переменная не существует, необходимо создать её).
	\item Скачать по адресу \url{https://s3-eu-west-1.amazonaws.com/artm/libs_win32_v110.7z} 
	статические библиотеки, необходимые для работы, и распаковать в папку \verb|libs| в корневой директории BigARTM. 
	\item Скачать по адресу
	\url{http://miru.hk/archive/ZeroMQ-4.0.3~miru1.0-x86.exe}
	библиотеку ZeroMQ, распаковать её, и присвоить системной переменной \verb|ZEROMQ32\_ROOT| путь к корневой директории ZeroMQ (если переменная не существует, необходимо создать её).
	\item Установить Python 2.7 (x32)
	\item Добавить корневую папку Python в переменную окружения \verb|PATH|.
	\item Следуйте инструкциям, описанным в файле
	
	\vspace{5pt}
	\verb|BigARTM_ROOT_DIRECTORY\3rdparty\protobuf\python\README.txt| 
\end{enumerate}

\subsection{Работа с библиотекой}

\paragraph{Запуск обучения} Основным классом, обеспечивающем пользовательский API, является \verb|MasterComponent|. Объект данного класса содержит в себе все созданные тематические модели и регуляризаторы, через него производится управление процессом обучения. 

Для того, чтобы запустить обучение тематической модели, требуется пошаговое выполнение следующей инструкции:

\begin{enumerate}
	\item Ядро библиотеки представляет собой скомпилированный модуль \verb|artm.dll|. Прежде всего требуется подключить его. Чтобы это сделать, необходимо поместить в программу следующие строки:
	
	\vspace{5pt}
	
	\verb|  address = os.path.abspath(os.path.join(os.curdir, os.pardir))| \\
	\verb|  os.environ['PATH'] = ';'.join([address + | \\
	\verb|    '\\Win32\\Release', os.environ['PATH']])| \\
	\verb|  library = ArtmLibrary(address + '\\Win32\\Release\\artm.dll')|
	
	\vspace{5pt}
	
	\item Необходимо создать объект класса \verb|MasterComponent|. Для этого нужно описать соответствующее proto-сообщение, которое (в базовой конфигурации) требует заполнения следующих полей:
	
	\vspace{5pt}
	
	\verb|  optional int32 processors_count = 2 [default = 1];| \\
	\verb|  optional string disk_path = 3;|
	
	\vspace{5pt}
	
	Первое поле представляет собой число процессоров, в рамках которых будет производится распараллеливание алгоритма. Второе поле описывает путь к папке, в которую будут сохранятся данные, преобразованные в \verb|Batches| (кроме того, именно в этой папке библиотека будет искать пакеты для своей работы). Создание и описание этого сообщения может иметь следующий вид:
	
	\vspace{5pt}
	
	\verb|  master_config = messages_pb2.MasterComponentConfig()| \\
	\verb|  master_config.processors_count = 1| \\
	\verb|  master_config.disk_path = 'disk_path'|	
	
	\vspace{5pt}
	
	После того, как конфигурационное сообщение сформировано, можно создать сам объект \verb|MasterComponent|:
	
	\vspace{5pt}
	
	\verb|  master_component = library.CreateMasterComponent(master_config)|
	
	\vspace{5pt}
	
	\item Следующим шагом требуется загрузить коллекцию и словарь для неё. Данное действие является техническим, в качестве базового примера можно использовать считывание, описанное в \verb|python_client.py|. 
	
	\item
	\label{step_1}
	 Теперь в \verb|master_component| нужно добавить тематические модели для обучения. Допустим, что требуется обучить одну модель. Для её создания необходимо заполнить соответствующее proto-сообщение. Оно будет выглядеть так:

	\vspace{5pt}

	\verb|  optional string name = 1 [default = ""];| \\
	\verb|  optional int32 topics_count = 2 [default = 32];| \\	
	\verb|  optional int32 inner_iterations_count = 4 [default = 10];| \\
	\verb|  repeated Score score = 7;|
	
	\vspace{5pt}
	
	Параметры имеют следующие назначения: 	\verb|name| --- имя тематической модели; \verb|topic_counts| --- число тем, которые будет искать в коллекции данная модель; третий  параметр назначает модели число внутренних итераций; в последнем поле указываются функционалы качества, которые наобходимо рассчитывать для данной модели в ходе её работы
	\footnote{На данный момент в библиотеке реализована только перплексия.}
	. Создать конфигурацию модели можно так:
	
	\vspace{5pt}
	
	\verb|  model_config = messages_pb2.ModelConfig()| \\
	\verb|  model_config.topics_count = 20| \\
	\verb|  model_config.inner_iterations_count = 10| \\	
	\verb|  score_ = model_config.score.add()| \\
	\verb|  score_.type = 0|
	        
	\vspace{5pt}
	
	\verb|score_.type = 0| соответствует добавлению модель требование подсчёта перплексии на каждой итерации. Теперь можно создать саму модель, отнеся её к созданному ранее объекту \verb|MasterComponent|, после чего активировать:
	
	\vspace{5pt}
	
	\verb|	model = master_component.CreateModel(master_component, model_config)| \\
	
	\vspace{5pt}
	
	\item 
	\label{step_2}
	После того, как модель была создана, необходимо запустить работу алгоритма
	\footnote{Вообще говоря, прежде, чем начинать счёт, в модель стоит добавить регуляризаторы. О том, как это сделать, подробно написано в соответствующем разделе.}
	. Для этого нужно описать цикл по числу внешних итераций (это число определяется пользователем), внутри которого должны быть такие строки кода:
	
	\vspace{5pt}
	
	\verb|  master_component.InvokeIteration(1)| \\
	\verb|  master_component.WaitIdle();|
	        
	\vspace{5pt}	
	
	Первая строка производит вызов одной внешней итерации работы алгоритма, вторая --- ожидание завершения выполнения это итерации.
	
	Этого достаточно для запуска алгоритма, однако в большинстве случаев требуется контролировать его работу (в т.ч. и следить за перплексией на данной итерации). Для этого следует выгрузить посчитанную на данный момент модель и просмотреть её параметры. Выгрузить модель можно, добавив ещё одну строку под предыдущими:
	
	\vspace{5pt}

	\verb|  topic_model = master_component.GetTopicModel(model)|
	        
	\vspace{5pt}		 
	
	Как было указано ранее, сама модель описывается proto-сообщением. Это сообщение имеет такой вид:

	\vspace{5pt}		 

	\verb|  optional string name = 1 [default = ""];| \\
	\verb|  optional int32 topics_count = 2;| \\
	\verb|  optional int32 items_processed = 3;| \\
	\verb|  repeated string token = 4;| \\
	\verb|  repeated FloatArray token_weights = 5;| \\
	\verb|  optional DoubleArray scores = 6;|	
	
	\vspace{5pt}
	
	Первые два поля были описаны ранее и имеют тот же смысл. Третье содержит число обработанных на данный момент документов (учитываются все проходы по каждому документу, включая повторные). В четвёртом поле лежит словарь для данной модели. В пятом --- веса каждого термина, полученные в результате работы алгоритма (матрица $\Phi$). Последнее поле содержит значения счётчиков на данной итерации (в описываемом примере --- значение перплексии). Данную информацию можно извлекать после каждой итерации и использовать для оценивания качества обучения.
	
	Выгруженная после последней внешней итерации тематическая модель является финальным результатом работы алгоритма.
	
	\item В случае необходимости перенастройки \verb|master_component| или \verb|model| используется функция \verb|Reconfigure()|:
	
	\vspace{5pt}

	\verb|  master_component.Reconfigure(new_master_config)| \\
	\verb|  model.Reconfigure(new_model_config)| 
	        
	\vspace{5pt}
	
	\item После окончания работы модели удалять её не нужно. Все модели будут автоматически удалены при вызове деструктора \verb|MasterComponent|.	
	
\end{enumerate}

{\bf Замечание:} Базовых типов \verb'DoubleArray' и \verb'FloatArray' в protocol buffers нет, это пользовательские тип, эквивалентные вещественному массиву (работать с ними нужно по тем же правилам, что и остальными сообщениями --- через создаваемый protocol buffers интерфейс).

{\bf Замечание:} Все функции типа \verb|Reconfigure()| являются не до конца тестированными, поэтому должны использоваться с осторожностью.

{\bf Замечание:} Пример пользовательского приложения над BigARTM можно найти в файле \verb|BigARTM\_ROOT\_DIRECTORY\src\python_client\python_client|. Выдержки кода, указанные в вышеописанной инструкции, заимствованы оттуда.

\subsection{Планируемые нововведения}

Ниже приведён список различных модификаций, которые появятся в release-версии библиотеки:

\begin{itemize}
	\item Кластерный параллелизм.
	\item Возможность работы в Linux.
	\item Интерфейсы на C++ и Java.
	\item Коллекция регуляризаторов.
	\item Поддержка 64-битной архитектуры.
	\item Усовершенствованный механизм хранения данных, необходимых для работы регуляризаторов.
	\item Коллекция функционалов качества.
	\item Хранение матриц $\Phi$ и $\Theta$ в разреженном виде.
\end{itemize}

\section{Регуляризаторы в BigARTM}\label{regularizers}

$\quad\;\:$Ключевым отличием BigARTM от других библиотек алгоритмов тематического моделирования является наличие регуляризаторов. Задача данного раздела --- описать API, предоставленный библиотекой для работы с существующими реализациями регуляризаторов, а также пояснить схему добавления новых.

\subsection{Пользовательский API}

$\quad\;\:$Как было описано ранее, для использования библиотеки требуется, в первую очередь, создать объект \verb'Instance', после чего добавить в него необходимые тематические модели, имеющие каждая свои конфигурации. Эти конфигурации, помимо всего прочего, содержат список имён регуляризаторов (подмножество списка, хранящегося в \verb'Instance').

Рассмотрим процесс добавления регуляризатора в \verb'Instance' и в конкретную модель по-подробнее.  Предположим, что в библиотеке уже описан регуляризатор типа \verb'MyRegularizer', который необходимо добавить в модель. Схема выполнения этой операции следующая:
\begin{enumerate}
	\item Открыть файл \verb'messages.proto', найти в нём protobuf-сообщение, соответствующее конфигурации регуляризатора типа \verb'MyRegularizer' (оно будет иметь имя \verb'MyRegularizerConfig').
	
	\item Используя язык, на котором описывается среда эксперимента, создать объект типа \verb'MyRegularizerConfig', заполнить необходимые поля теми параметрами регуляризатора, которые требуются. 
	
	\item Создать объект типа \verb'RegularizerConfig', который является обёрткой для регуляризатора любого типа. У этого объекта имеется три поля, которые требуется заполнить --- имя регуляризатора, его тип (найти тип, соответствующий \verb'MyRegularizer' можно в \verb'Type' в protobuf-сообщении, описывающем \verb'RegularizerConfig') и конфигурацию регуляризатора \verb'MyRegularizer'. Следует учесть, что это поле требует в качестве значения строку --- сериализованное сообщение \verb'MyRegularizerConfig', а не сам объект, созданный в п.2, поэтому его следует сериализовать, вызвав 
	
	\vspace{10pt}
	\verb'MyRegularizerConfig.SerializeToString()'.
	\vspace{10pt}
	
	\item Подготовив конфигурацию для \verb'MyRegularizer', добавим его в \verb'Instance', вызвав функцию
	
	\vspace{10pt}
	\verb|regularizer_object = CreateRegularizer(instance, regularizer_config_wrapper)| 
	\vspace{10pt}
	
	где \verb'instance' --- номер объекта \verb'Instance', а \verb'regularizer_config_wrapper' --- объект-обёртка типа \verb'RegularizerConfig'.
	
	\item После того, как регуляризатор был создан, а его имя было добавлено в список существующих для данного \verb'Instance' регуляризаторов, требуется указать имя этого регуляризатора в соответствующих списках тех моделей, к которым его нужно применять. Для этого нужно добавить в сообщение-конфигурацию модели, в поле \verb'regularizer_name', новый элемент, соответствующий имени регуляризатора. После требуется обновить конфигурацию модели.
	
	\item В случае, когда требуется заменить конфигурацию существующего объекта регуляризатора, необходимо модифицировать соответствующим образом объект-конфигурацию, после чего вызвать функцию 
	
	\vspace{10pt}
	\verb|regularizer_object.Reconfigure(regularizer_config_wrapper)| 
	\vspace{10pt}
	
	\item Удаление регуляризатора из модели производится по тому же принципу, что и добавление --- необходимо удалить имя этого регуляризатора из соответствующего списка в параметрах модели и реконфигурировать её. Удаление объекта регуляризатора из \verb'Instance' произойдёт при вызове \verb'__exit__()' (в C++ --- при вызове деструктора).
			
\end{enumerate}

Регуляризаторы матрицы $\Theta$, будучи добавленными в модель, будут вызываться автоматически при работе алгоритма. Вызов регуляризаторов $\Phi$ производится пользователем (например, после одного прохода по коллекции), для этого используется функция

	\vspace{10pt}
	\verb|model.InvokePhiRegularizers()| 
	\vspace{10pt}
	
\noindent где \verb'model' --- модель, для которой производится регуляризация матрица $\Phi$.

Регуляризаторы для матрицы Phi имеют один набор параметров для каждой внешней итерации (прохода по коллекции), менять этот набор, соотвественно, можно по завершении текущей итерации. Для регуляризаторов матрицы $\Theta$ также можно задавать новый набор параметров каждую внешнюю итерацию. Однако следует помнить, что регуляризаторы этой матрицы могут требовать разные параметры на этапе внутренней итерации (прохода по документу), и редактировать эти параметры во время внешней итерации нельзя. Поскольку число внутренних итераций задаётся до начала работы алгоритма, предполагается, что конфигурация регуляризатора для $\Theta$ будет содержать по одному набору параметров для каждой внутренней итерации до начала соответствующей внешней.

\subsection{Создание нового регуляризатора}

$\quad\;\:$Опишем процесс создания нового регуляризатора и добавления его в библиотеку. Данные действия подчиняются некоторым общим правилам, поэтому изложим их в виде последовательности шагов:

\begin{enumerate}
	\item В первую очередь требуется описать в файле \verb'messages.proto' protobuf-сообщение ---  конфигурацию нового регуляризатора, после чего добавить соответствующий элемент в поле \verb'Type' в общем сообщении \verb'RegularizerConfig'.
	
	\item Затем необходимо написать \verb'.h' и \verb'.cc' файлы создаваемого регуляризатора и добавить их в проект \verb'libartm'. В данном пункте следует учитывать несколько деталей, на которых остановимся подробнее в конце этого подраздела.
	
	\item После добавить в файл \verb'c_interface.cc #include' с именем \verb'.h' файла нового регуляризатора.
	
	\item В  \verb'c_interface.cc' нужно найти функцию \verb'ArtmReconfigureRegularizer()'. В ней есть switch по типам регуляризаторов, в который требуется добавить case с типом своего регуляризатора (для этого можно просто скопировать один из существующих существующий case и поменять в нём имена регуляризатора и его конфигурации).
	
	\item Для того, чтобы использовать добавленный регуляризатор, осталось скомпилировать \verb'messages.proto' (используя компилятор \verb'protoc') в файлы на C++ и Python, после чего перекомпилировать проекты \verb'libartm' и \verb'artm'.
\end{enumerate}

\paragraph{Рекомендации по написанию файлов регуляризаторов.} Любой регуляризатор, описанный в библиотеке, должен удовлетворять нескольким требованиям:

\begin{itemize}
	\item Регуляризатор обязательно пишется (как и всё ядро библиотеки) на С++ (.cс и .h файлы).

	\item Регуляризатор представляет собой класс-наследник класса \verb'RegularizerInterface'. Данный класс содержит два метода
	
	\vspace{10pt}
	\verb|bool RegularizeTheta(const Item& item,| 
	
	\verb|    std::vector<float>* n_dt, int topic_size, int inner_iter)|
	
	\verb'bool RegularizePhi(TopicModel* topic_model)'
	\vspace{10pt}
	
	Как понятно из названия, каждый метод отвечает за регуляризацию соответствующей матрицы.
	
	\item Класс регуляризатора вместе со всем своим содержимым должен быть описан в namespace \verb'regularizer'.
	
	\item Технически допустима реализация в одном классе регуляризатора сразу для обеих матриц, однако рекомендуется этого избегать и описывать один (в математическом смысле) регуляризатор в двух разных классах и работать с ними как с разными объектами. Это существенно упростит конфигурацию отдельного регуляризатора, использование его данных в функциях регуляризации, а также избавит от необходимости задавать ненужные данные, когда потребуется регуляризация только одной матрицы.
	
	\item Файл \verb'.h' будет отличаться для разных регуляризаторов только названием, достаточно скопировать какой-нибудь существующий (регуляризирующий ту же матрицу) и поменять необходимые имена типов.
	
	\item Файл \verb'.cc' будет включать в себя реализацию соответствующего метода регуляризации. Сама реализация, в свою очередь, состоит из трёх концептуальных этапов --- считывания данных из объекта-конфигурации, проверки корректности этих данных и собственно регуляризации. Считывание данных рекомендуется производить в структуры тех же типов, что и используемые в конфигурации --- т.е. описанные в файлах \verb'messages.pb.h' и \verb'messages.pb.cc' protobuf-типы. Во время проверки корректности ввода следует возвращать из функции \verb'false' в случае выявления несоответствия. Наконец, после окончания этапа регуляризации следует вернуть \verb'true'.
	
\end{itemize}

В целом, при написании нового регуляризатора рекомендуется опираться на существующие реализации, это может сэкономить немало времени.

\subsection{Регуляризатор сглаживания/разреживания}

$\quad\;\:$Базовым регуляризатором в BigARTM является т.н. регуляризатор сглаживания/разреживания. Он представляет собой модификацию сглаживающего регуляризатора Дирихле (встроенного в модель LDA). Несмотря на простоту, уже данный регуляризатор решает задачи, более сложные, чем регуляризатор Дирихле. Именно на основе этого регуляризатора будут строится все эксперименты в рамках данной работы, поэтому рассмотрим его поподробнее\footnote{Описание регуляризатора Дирихле можно найти в \cite{voron2013ptm}.}.

\paragraph{Теоретическое описание}\footnote{Подробное описание и лингвистические обоснования этого регуляризатора можно найти в \cite{voron_potap_14}}
Разделим множество тем $T$ на предметные $S$ и фоновые $B$. Зададим вектор-столбцы параметров таким образом, чтобы в обеих матрицах $\Phi$ и $\Theta$ темы из $S$ одинаково разреживались, а темы из $B$ --- сглаживались каждая собственным образом. Это приводит к тому, что 

\begin{enumerate}
	\item каждая тема из $S$ приобретает некоторое выраженное ядро терминов, отличающее её от остальных тем;
	\item каждая фоновая тема из $B$ сглаживается, причём различность параметров приводит к тому, что разные темы выполняют различные задачи (выделение стоп-слов, общей лексики коллекции и т.п.).
\end{enumerate}

\paragraph{Детали программной реализации}
Данный регуляризатор был реализован в соответствии с инструкцией из предыдущего подраздела. Регуляризатор был разделён на два: для $\Theta$ и для $\Phi$. Далее мы будем рассматривать их как два разных объекта: \verb'SmoothSparseTheta' и \verb'SmoothSparsePhi'. 

Конфигурационные protobuf-сообщения имеют следующий вид:

\vspace{5pt}
\noindent\verb|message SmoothSparseThetaConfig {| \\
\verb|    repeated double alpha_0 = 1;| \\
\verb|    repeated DoubleArray tilde_alpha = 2;| \\
\verb|}|

\noindent\verb|message SmoothSparsePhiConfig {| \\
\verb|    required int32 background_topics_count = 1;| \\
\verb|    required double beta_0 = 2;| \\
\verb|    required DoubleArray tilde_beta = 3;| \\
\verb|    repeated double background_beta_0 = 4;| \\
\verb|    repeated DoubleArray background_tilde_beta = 5;| \\
\verb|}|

Для регуляризации $\Theta$ требуется на каждую итерацию прохода по документу один набор параметров. Это набор состоит из вектора длиной с число тем, которые нужны выделить, и коэффициента при этом векторе. После поэлементного перемножения данный вектор должен содержать отрицательные числа для тем из $S$, и положительные --- для тем из $B$.

Регуляризатор для $\Phi$ требует в явном виде указать число фоновых тем (это нужно для проверки корректности параметров). \verb'tilde_beta' --- вектор длиной с текущее число терминов в матрице $\Phi$, предназначенный для разреживания тем из $S$. \verb'beta_0' --- коэффициент при нём. Последние два параметра имеют тот же смысл, только они предназначены для фоновых тем, по одному набору на каждую.  

{\bf Замечание:} Базового типа \verb'DoubleArray' в protocol buffers нет, это пользовательский тип, эквивалентный вещественному массиву.


\section{Эксперименты}\label{experiments}

\subsection{Описание экспериментов}

В данном разделе будет рассмотрен модельный эксперимент с использованием регуляризаторов в BigARTM. Его задачей является демонстрация использования реализованного механизма регуляризаторов, от него не ожидается получения ответов на какие-либо исследовательские вопросы.

\subsubsection{Описание эксперимента} 
Производится обучение двух тематических моделей --- с регуляризатором и без. Для регуляризации модели используется регуляризатор сглаживания/разреживания. Параметры эксперимента следующие:

\begin{enumerate}
	\item Обучающая коллекция документов --- NIPS ($\approx$ 1600 документов).
	\item Объём словаря --- $\approx$ 13000 терминов.
	\item Число внешних итераций --- 12, внутренних --- 10.
	\item Число процессоров --- 2.
	\item Общее число тем --- 18, фоновых тем --- 3 (для модели без регуляризатора все темы равнозначны).
\end{enumerate} 

Регуляризация имеет следующую траекторию:
\begin{itemize}
	\item 1 --- 4 итерации: 
	\item 5 --- 8 итерации: 
	\item 9 --- 12 итерации:  
\end{itemize}
\marginpar{ToDO}

\subsubsection{Результаты эксперимента}

Сравнение качества моделей производилось по перплексии на обучающей выборке. Соответствующие графики построены на рис ... Из них видно, что регуляризация влияет на конечный результат. То, что это влияние не столь существенно --- вина неаккуратной настройки параметров модели и регуляризатора (они неоптимизированные, не учитывают текущего состояния модели). Кроме того, перплексия на обучающей выборке --- не лучший функционал качества. Существенно более важны такие параметры, как различность тем, разреженность матриц $\Phi$ и $\Theta$ (без учёта фоновых тем), объём ядер тем (т.е. слов, характеризующих данную тему). Регуляризация имеет перед собой задачу оптимизировать именно эти величины, не ухудшив при этом переплексию по сравнению с нерегуляризованной моделью.

\section{Заключение}\label{results}
$\quad\;\:$Все задачи, поставленные перед данной курсовой работой, были выполнены. Был произведён обзор текущих возможностей BigARTM, перспективы её развития, описаны инструкции по работе с библиотекой. Кратко были рассмотрены существующие библиотеки тематического моделирования, их алгоритмические и архитектурные особенности, в т.ч. и нашедшие применение в BigARTM. 

В библиотеке BigARTM была создана и отлажена структура работы с регуляризаторами тематических моделей. Она была протестирована и доказала свою работоспособность. Были описаны примеры регуляризаторов, которые могут быть использованы как при решении реальных задач, так и в качестве основы для создания более сложных механизмов регуляризации.

Дальнейшая деятельность в рамках проекта по созданию библиотеки будет направлена на внедрение описанных в тексте работы планируемых нововведений.


\newpage
\addcontentsline{toc}{section}{Список литературы}
\begin{thebibliography}{00}

\bibitem{voron2014}
Воронцов К. В. Аддитивная регуляризация тематических моделей коллекций текстовых документов // Доклады РАН. 2014. — Т. 455., №3. 268–271.

\bibitem{hofmann_plsa}
T.~Hofmann Probabilistic latent semantic indexing // Proceedings of the 22nd annual international
ACM SIGIR conference on Research and development in information retrieval. — New York, NY,
USA: ACM,~1999. — Pp. 50–57.

\bibitem{protobuf} 
https://developers.google.com/protocol-buffers/docs/tutorials

\bibitem{ad_lda}
David Newman, Arthur Asuncion, Padhraic Smyth and Max Welling --- Distributed Algorithms for Topic Models, 2009.

\bibitem{plda}
Yi Wang, Hongjie Bai, Matt Stanton, Wen-Yen Chen and Edward Y.Chang --- PLDA: Parallel Latent Dirichlet Allocation for Large-Scale Applications, 2009.

\bibitem{y_lda}
A. J. Smola and S. Narayanamurthy --- An architecture for parallel topic models, 2010.

\bibitem{plda_plus}
Liu, Z., Zhang, Y., Chang, E. Y., and Sun, M. --- PLDA+: Parallel Latent Dirichlet Allocation with Data Placement and Pipeline Processing, 2011.

\bibitem{voron_potap_14} 
Воронцов~К.В., Потапенко~А.А. --- Аддитивная регуляризация тематических моделей,~2014.

\end{thebibliography}

\end{document}
