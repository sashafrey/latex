$\quad\;\:$Все задачи, поставленные перед данной курсовой работой, были выполнены. Был произведён обзор текущих возможностей BigARTM, перспективы её развития, описаны инструкции по работе с библиотекой. Кратко были рассмотрены существующие библиотеки тематического моделирования, их алгоритмические и архитектурные особенности, в т.ч. и нашедшие применение в BigARTM. 

В библиотеке BigARTM была создана и отлажена структура работы с регуляризаторами тематических моделей. Она была протестирована и доказала свою работоспособность. Были описаны примеры регуляризаторов, которые могут быть использованы как при решении реальных задач, так и в качестве основы для создания более сложных механизмов регуляризации.

Дальнейшая деятельность в рамках проекта по созданию библиотеки будет направлена на внедрение описанных в тексте работы планируемых нововведений.