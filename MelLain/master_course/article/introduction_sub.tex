
$\quad\;\:$Тематическое моделирование --- активно развивающаяся в последние годы область машинного обучения. Оно позволяет решать задачи тематического поиска, категоризации и кластеризации корпусов текстовых документов. Аналогичные задачи решаются для коллекций изображений и видеозаписей.

Тематическая модель определяет, к каким темам относится каждый документ, а также то, какие термины из словаря образуют ту или иную тему.

В работе \cite{voron2014} предлагается полувероятностный подход к тематическому моделированию --- аддитивная регуляризация тематических моделей (АРТМ). Он позволяет записать любое количество дополнительных требований к тематической модели в виде взвешенной суммы критериев, добавляемых к основному функционалу логарифмированного правдоподобия. В \cite{voron2014} показано, что многие известные тематические модели допускают такое представление, то есть фактически являются лишь формой регуляризации. При этом, в отличие от стандартных задач машинного обучения --- классификации и регрессии, в тематическом моделировании возникает огромное разнообразие регуляризаторов, направленных на улучшение модели и учёт дополнительной информации о текстовой коллекции. Задача тематического моделирования является по сути многокритериальной, и АРТМ позволяет выразить это непосредственным образом. Более распространённый в литературе байесовский подход основан на гипотезе о существовании адекватной вероятностной модели порождения текста. Эта гипотеза представляется избыточно сильной, так как далеко не все лингвистические требования и знания о естественном языке допускают вероятностую трактовку. Кроме того, техника байесовского вывода, как показывают работы последних лет, создаёт большое число технических трудностей при построении комбинированных моделей и попытках одновременного учёта большого числа разнородных требований.

Благодаря АРТМ появляется новая возможность --- создать библиотеку из десятков различных регуляризаторов и строить решения прикладных задач путём обоснованного выбора  подмножества регуляризаторов. В данной курсовой работе рассматривается архитектура и отдельные элементы библиотеки BigARTM, в которой в настоящее время реализуется данная концепция. Особое внимание уделяется мультипроцессорному и кластерному распараллеливанию, поскольку библиотека BigARTM изначально проектируется для тематического моделирования больших коллекций.   

Работа имеет следующую структуру: в разделе \ref{defenitions} введены базовые обозначения и определения, необходимые для дальнейшего изложения; алгоритм обучения тематических моделей и идея регуляризации вводятся в разделе \ref{learning}; в разделе \ref{overview} приводится краткий обзор существующих параллельных библиотек тематического моделирования; раздел \ref{library} описывает общую архитектуру библиотеки и схему пользовательского взаимодействия с BigARTM; раздел \ref{regularizers} посвящён реализации и использованию регуляризаторов в библиотеке; в \ref{experiments} разделе показаны эксперименты с регуляризаторами; раздел \ref{results} предназначен для выводов и подведения итогов курсовой работы.