
\subsection{PLSA}

Вероятностный латентный семантический анализ (PLSA) был предложен Т.Хофманном в~\cite{hofmann_plsa}.

Примем гипотезу условной независимости, утверждающую, что вероятность появления термина в данном документе зависит только от темы этого термина и не зависит от документа. Вероятностная порождающая модель PLSA имеет следующий вид: 
\[
	p(w|d) = \sum_{t \in T} p(w|t) p(t|d)
\]

PLSA можно реализовать с помощью ЕМ-алгоритма. Итерационный процесс состоит из двух шагов --- Е-шага (Expectation) и М-шага (Maximization). На Е-шаге по текущим значениям $\phi_{wt}$ и $\theta_{td}$ c помощью формулы Байеса вычисляются условные вероятности $p(t|d,w)$:
\[
	H_{dwt} = p(t|d,w) = \cfrac{\phi_{wt}\theta_{td}}{\sum_{s \in T}\phi_{ws}\theta_{sd}}
\]

На М-шаге по условным вероятностям $H_{dwt}$ вычисляются новые приближения параметров $\phi_{wt}$ и $\theta_{td}$. Используются указанные в предыдущем разделе формулы:
\[
	\phi_{wt} = \cfrac{n_{wt}}{n_t}, \quad
	\theta_{td} = \cfrac{n_{dt}}{n_d}, \quad	
\]

\subsection{Аддитивная регуляризация}

Неоднозначность матричного разложения $F \approx \Theta \Phi$ даёт свободу выбора матриц из правой части равенства, позволяя наложить на тематическую модель дополнительные требования.  
Модифицируем максимизируемый функционал \ref{eq_1}:

\begin{equation}
	\quad L(\Phi,\Theta) + R(\Phi,\Theta) \rightarrow \max_{\Phi,\Theta}
\end{equation}	


\[
 	R(\Phi,\Theta) = \sum_{i = 1}^{n} \tau_i R_i(\Phi,\Theta)
\]	
где $R_i(\Phi,\Theta)$ --- дополнительные требования к модели, $\tau_i$ --- неотрицательные  коэффициенты регуляризации, выполнены условия неотрицательности и нормировки столбцов матриц $\Phi$ и $\Theta$.
 	 
Решение этой задачи приводит к обощению формул М-шага в ЕМ-алгоритме:
\begin{equation}
	\phi_{wt} = \cfrac{\left(n_{wt} + \phi_{wt} \cfrac{\partial R}{\partial \phi_{wt}} (\Phi,\Theta) \right)_+}{\sum_{u \in W} \left(n_{ut} + \phi_{ut} \cfrac{\partial R}{\partial \phi_{ut}} (\Phi,\Theta) \right)_+}, \quad 
 	\theta_{td} = \cfrac{\left(n_{dt} + \theta_{td} \cfrac{\partial R}{\partial \theta_{td}} (\Phi,\Theta) \right)_+}{\sum_{s \in T} \left(n_{ds} + \theta_{sd} \cfrac{\partial R}{\partial \theta_{sd}} (\Phi,\Theta) \right)_+}
\end{equation} 
 	 
 	 $n_{wt}$ и $n_{dt}$ определяются аналогично из формул предыдущего раздела.
 	 
Таким образом, суть добавления регуляризаторов --- в простом изменении формул М-шага.

{\bf Замечание:} Использование регуляризаторов требует аккуратного выстраивания т.н. траектории регуляризации. Этот процесс включает в себя настройку параметров, определение времени подключения/отключения того или иного регуляризатора, используемого в модели и т.п.