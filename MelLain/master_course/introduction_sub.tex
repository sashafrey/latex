
$\quad\;\:$Тематическое моделирование --- активно развивающаяся в последние годы область машинного обучения. Оно позволяет решать задачи тематического поиска, категоризации и кластеризации корпусов текстовых документов. Аналогичные задачи решаются для коллекций изображений и видеозаписей.

Тематическая модель определяет, к каким темам относится каждый документ, а также то, какие термины из словаря образуют ту или иную тему.

Одной из проблем существующих алгоритмов обучения является невозможность совмещения большого числа функциональных требований в одной модели. LDA, который является на сегодняшний день стандартом в данной области, не позволяет комбинировать более 2-3 требований из-за сложности используемого математического аппарата.

В работе \cite{voron2013ptm} рассматривается теория аддитивной регуляризации тематических моделей (АРТМ), построенная над более простым алгоритмом обучения PLSA, и позволяющая решить описанную проблему. Библиотека BigARTM представляет собой программную реализацию этой концепции, адаптированную под мультипроцессорное и кластерное распараллеливание.

Список основных задач данной курсовой работы: произвести обзор BigARTM, реализовать в ней  механизм добавления/удаления регуляризаторов, описать возможности для его использования и модификации, оценить качество работы. 

{\bf Замечание:} Поскольку библиотека в настоящий момент находится в стадии разработки, описанные возможности и API являются текущими и не отражают полной картины того, как будет выглядеть BigARTM в своей release-версии.