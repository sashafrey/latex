
Тематическое моделирование --- активно развивающаяся в последние годы область машинного обучения. Оно позволяет решать задачи тематического поиска, категоризации и кластеризации корпусов текстовых документов. Аналогичные задачи решаются для коллекций изображений и видеозаписей.

Тематическая модель определяет, к каким темам относится каждый документ, а также то, какие термины из словаря образуют ту или иную тему.

Одной из проблем существующих алгоритмов обучения является невозможность совмещения большого числа функциональных требований в одной модели. LDA, который является на сегодняшний день стандартом в данной области, не позволяет комбинировать более 2-3 требований из-за сложности используемого математического аппарата.

В работе \cite{voron2013ptm} рассматривается теория аддитивной регуляризации тематических моделей (АРТМ), построенная над более простым алгоритмом обучения PLSA и позволяющая решить описанную проблему. Библиотека BigARTM представляет собой программную реализацию этой концепции, адаптированную под мультипроцессорное и кластерное распараллеливание.

Список основных задач данной курсовой работы: произвести обзор BigARTM, реализовать в ней  механизм добавления/удаления регуляризаторов, описать возможности его использования и модификации. 

Работа имеет следующую структуру: в разделе \ref{defenitions} введены базовые обозначения и определения, необходимые для дальнейшего изложения; алгоритм обучения тематических моделей и идея регуляризации вводятся в разделе \ref{learning}; в разделе \ref{overview} приводится краткий обзор существующих параллельных библиотек тематического моделирования; раздел \ref{library} описывает общую архитектуру библиотеки и схему пользовательского взаимодействия с BigARTM; раздел \ref{regularizers} посвящён реализации и использованию регуляризаторов в библиотеке; в \ref{experiments} разделе показаны эксперименты с регуляризаторами; раздел \ref{results} предназначен для выводов и подведения итогов курсовой работы.