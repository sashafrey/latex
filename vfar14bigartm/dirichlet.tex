\documentclass[]{article}

% TEMP PACKAGES!
\usepackage[utf8]{inputenc}
\usepackage[russian]{babel}
\usepackage{amsmath}
\usepackage{cmap}

% These ones should be added to the article's .tex file...
\usepackage{tikz}
\usepackage{pgfplots}
\usepackage{verbatim}
\usepackage{standalone}
\usepackage{array}
\usepackage{amsmath,amssymb}
\usepackage{fullpage}

\usepackage{url}
\renewcommand{\vec}[1]{{\boldsymbol #1}}

\begin{document}

\section{Распределение Дирихле.}
Смотрим на Википедии: \url{http://en.wikipedia.org/wiki/Dirichlet_distribution}.

Если случайная величина имеет распределение Дирихле с вектором параметров $\vec{\alpha}$, то вот ее мода и среднее:

\begin{gather}
\vec{x} \sim \text{Dir}(\vec{\alpha}), \quad \vec{\alpha} \in \mathbb{R}^K \\
x^\text{mode}_i = \frac{\alpha_i - 1}{\sum_{j} \alpha_j - K}, \, \alpha_i > 1 \\
x^\text{mean}_i = \frac{\alpha_i}{\sum_{j} \alpha_j}
\end{gather}

В случае, если $\alpha_i \leq 1$, то с модой распределения Дирихле получается хитрая вещь. Пояснение можно почитать вот здесь: \url{http://www.quora.com/Why-can-hyper-parameters-in-beta-and-dirichlet-distribution-be-less-than-1}.

Посмотрим на плотнсть распределения Дирихле:
\begin{gather}
	p(\vec{x}) = \frac{1}{B(\vec{\alpha})} \prod_{i=1}^K x_i^{\alpha_i-1} \\
	B(\vec{\alpha}) = \frac{\prod_{i=1}^K \Gamma(\alpha_i)}{\Gamma(\sum_{i=1}^K \alpha_i)}
\end{gather}

Если $\alpha_i = 1$, то в плотность $p(\vec{x})$ вероятность $i$-того события $x_i$ не входит: $x_i^{\alpha_i-1} = 1$. Распределению равномерно, какое значение будет принимать $x_i$, поэтому мода по этой координате не определена.

Если $0 < \alpha_i < 1$, то при фиксированных всех вероятностях событий, кроме $x_i$, плотность зависит следующим образом:
\begin{gather}
	p(\vec{x}) \propto \frac{1}{x_i^{1 - \alpha_i}}
\end{gather}
Это значит, что чем ближе вероятность $i$-того события к 0 — тем вероятнее такое распределение с точки зрения Дирихле. Но формально моды не существует, поскольку в $x_i = 0$ эта функция плотности не определена.

Если представить, что функция плотности в точке $x_i = 0$ равна $+\infty$ и это не мешает плотности интегрироваться в 1, то мода вполне себе существует и легко показать, что в таком случае:
\begin{gather}
x^\text{mode}_i \propto \max\left(0, \alpha_i - 1 \right).
\end{gather}


\section{Формулы для перевода модели LDA в $\Phi, \Theta$}

$$\phi_t \sim \text{Dir}(\lambda_t) \quad \Rightarrow \quad \begin{cases}
  \phi_{w,t}^\text{map} \propto (\lambda_{w,t} - 1)_{+} \\
  \phi_{w,t}^\text{mean} \propto \lambda_{w,t}
\end{cases}$$

$$\theta_d \sim \text{Dir}(\gamma_d) \quad \Rightarrow \quad \begin{cases}
  \theta_{d,t}^\text{map} \propto (\gamma_{d,t} - 1)_{+} \\
  \theta_{d,t}^\text{mean} \propto \gamma_{d,t}
\end{cases}$$

\end{document}