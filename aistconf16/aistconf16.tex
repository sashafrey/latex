\documentclass[russian,english]{llncs}
\usepackage[utf8]{inputenc}
\usepackage[T2A]{fontenc}
\usepackage[final]{graphicx}
\usepackage{epstopdf}
\usepackage[labelsep=period]{caption}
\usepackage[hyphens]{url}
\usepackage{amssymb,amsmath,mathrsfs}
\usepackage[russian,english]{babel}
%\usepackage{multicol}
\usepackage[ruled,vlined,linesnumbered,algo2e]{algorithm2e}
%\usepackage{algorithm}
%\usepackage[noend]{algorithmic}
\usepackage{color}
\usepackage{cmap}
\usepackage{array}
\usepackage{tikz}
\usepackage{pgfplots}
%\usepackage{verbatim}
\usepackage{standalone}

\tolerance=1000
\hbadness=5000
\newcommand{\const}{\mathrm{const}}
\newcommand{\tsum}{\mathop{\textstyle\sum}\limits}
\newcommand{\tprod}{\mathop{\textstyle\prod}\limits}
\newcommand{\cov}{\mathop{\rm cov}\limits}
\newcommand{\Dir}{\mathop{\rm Dir}\nolimits}
\newcommand{\norm}{\mathop{\rm norm}\limits}
\newcommand{\KL}{\mathop{\rm KL}\nolimits}
%\renewcommand{\geq}{\geqslant}
%\renewcommand{\leq}{\leqslant}
\newcommand{\eps}{\varepsilon}
\newcommand{\cond}{\mspace{3mu}{|}\mspace{3mu}}
\newcommand{\Loss}{\mathscr{L}}
\newcommand{\RR}{\mathbb{R}}
\newcommand{\cL}{\mathscr{L}}
\newcommand{\cP}{\mathscr{P}}
\newcommand{\kw}[1]{\textsf{#1}}
\SetKwFor{ForAll}{\textbf{for all}}{}{}

%... and these rows too.
\pgfplotsset{ every non boxed x axis/.append style={x axis line style=-},
     every non boxed y axis/.append style={y axis line style=-}}
\pgfplotsset{compat = 1.3}

\begin{document}
%%Analysis of Images, Social Networks, and Texts
\title{
    Panda: Parallel Non-blocking Deterministic Algorithm for Online Topic Modelling
}
\author{
    Oleksandr Frei\inst{1}
    \and
    Murat Apishev\inst{2}
    \and
    Peter Romov\inst{3}
}
\institute{\noindent
    Schlumberger Information Solutions,
    ~~\email{oleksandr.frei@gmail.com}
    \and
    Lomonosov Moscow State University,
    ~~\email{great-mel@yandex.ru}
    \and
    Yandex,
    Moscow Institute of Physics and Technology,
    ~~\email{peter@romov.ru}
}

\maketitle

\begin{abstract}
Deterministic behavior is an important property for a stochastic algorithm
because it allows to reproduce the results from run to run.
In concurrent implementation it might not be sufficient to
just fix a random seed, because threads scheduling can affect the result.
In this paper we present a deterministic modification of parallel non-blocking algorithm
for online topic modeling, previously developed for BigARTM library.
We implement new algorithm in BigARTM and demonstrate that new version converges faster
than previous algorithm in terms of perplexity.

\vspace{1em}
\textbf{Keywords:}
    probabilistic topic modeling,
    Probabilistic Latent Sematic Analysis,
    Latent Dirichlet Allocation,
    Additive Regularization of Topic Models,
    stochastic matrix factorization,
    EM-algorithm,
    BigARTM.
\end{abstract}

\section{Introduction}

The rest of the paper is organized as follows.
In~section~\ref{sec:Previous}
we introduce basic notation and summarize previous algorithm for online topic modeling, used in $BigARTM v0.6$.
In~section~\ref{sec:Algorithm}
we~present a deterministic non-blocking modification of the algorithm.
In~section~\ref{sec:Architecture}
we~describe the architecture, implemented in $BigARTM v0.7$.
In~section~\ref{sec:Experiments}
we~report results of our experiments on large datasets.
In~section~\ref{sec:Conclusions}
we~discuss advantages, limitations and open problems of BigARTM.

TBD: decide whether we should discuss really technical topics:
\begin{itemize}
    \item Details of CLI interface and python interface, usage examples
    \item List new features: coherence score and regularizer, classification, documents markdown (aka $p_{tdw}$ matrices)
    \item Our technologies (Protobuf for low-level C API, Boost Serialize for import/export, GLog, GFlags, GTest, etc)
    \item Our build solution and CI solution (CMake, Visual Studio, GitHub, Git submodules, Travis, Apveyour, Read-The-Docs)
\end{itemize}

\section{Previous algorithm}
\label{sec:Previous}

\section{Deterministic algorithm}
\label{sec:Algorithm}

\section{New architecture}
\label{sec:Architecture}

\section{Experiments}
\label{sec:Experiments}

\section{Conclusions}
\label{sec:Conclusions}

TBD

\bigskip
\subsubsection*{Acknowledgements.}

TBD

%%%%%%%%%%%%%%%%%%%%%%%%%%%%%%%%%%%%%%%%%%%%%%%%%%%%%%%%%%%%%%%%%%%%%%%%%%%%
%\bibliographystyle{splncs03}
%\bibliography{MachLearn}

\begin{thebibliography}{10}
%\providecommand{\url}[1]{\texttt{#1}}
%\providecommand{\urlprefix}{URL }

TBD

\end{thebibliography}

\end{document}

